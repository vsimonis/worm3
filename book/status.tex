
%%% Local Variables: 
%%% mode: latex
%%% TeX-master: t
%%% End: 

\documentclass{article}
\usepackage{spverbatim}
\begin{document}
\title{Nematodes: To do}
\maketitle

\setcounter{enumi}{0}
\setcounter{section}{0}
\section{General}
\begin{itemize}
\item Plan is to have 3 trackers that can work concurrently.
\end{itemize}


\section{May 9th}
\subsection{Poster}
\begin{itemize} 
\item 36x48
\item Start from colloquium presentation
\item narrow down Hoky's research question
\item Tracker: humans can't observe worms for 45 min and record and quantify their movement with the precision and detail required to provide evidence supporting a hypothesis. We wish to record nematodes moving (specifically wild type off food and candidate mutants on food) and identify behavior sequences. These behaviors are already well-defined.
\item Knowing your audience: In general, people at the session will have neither Cell and Molecular Biology background nor a Computer Science background. These will be faculty members from DPU and RFUMS working on a range of projects including: Social Health, Nursing, etc...
\item In this light, any technical information provided (from either side) needs to be well explained, accessible yet engaging and interesting. 
\end{itemize}




\subsubsection{Research questions}
\begin{enumerate}
\item \textbf{Which motor neurons cause \textit{off-food} behavior of \textit{C. elegans}?}
\item Hypothesis: Certain mutants on food behave like wild type \textit{C. elegans} off food.
\item Identify the neurotransmitters and neurons recognizing absence of food
\end{enumerate}



\subsection{``Live'' Demo}
\begin{itemize}
\item Where EXACTLY is this poster session taking place? Depending on the environment we will have a live worm or play the Tracker from a video.
\item Regardless, plan is to get a \textit{good} video on May 2nd.
\item Will use 15cm plate
\item Should demonstrate ability to track long term. 
\item ``Oh and by the way, can the tracker be done in 2 weeks?''
\end{itemize}


\subsubsection{Functional improvements}
\begin{itemize}
\item Plan for a normal start sequence
\item Have a toggle for Tracking \verb|on| vs \verb|off|: this relates to the initialization sequence to quickly start tracking once the nematode is found. 
\item Make a decision about which camera to use moving forward. It has been suggested to use both. 
\item Expect to have to tweak some parameters.; 
\item Choose a resolution
\end{itemize}

\subsubsection{Cosmetic improvements}
\begin{itemize}
\item Currently, \verb|debugWindow| shows the Gaussian blur. May be more interesting to have the \verb|liveWindow| show raw feed and the \verb|debugWindow| show the boxes and dots. 
\item Have some indication of whether recording is happening or not
\end{itemize}

\section{Status}
\subsection{Errors}
\begin{table}[h]
\begin{tabular}{l | l | p{8cm} }
File & Line num & Desc\\ \hline
\verb|tracker.py| & 161 (\verb|shutDown()|) &\verb|File annotationH299.csv does not exist.| Leftover from surf/sift programs. Take this out\\
\verb|finder.py|  & 80 (\verb|writeOut()|) & [blank]\\ \hline
\end{tabular}
\caption{Current errors to fix}
\label{tab:errors}
\end{table}

\subsection{Modes to run in}

\subsection{class WormFinder}
\subsubsection{Methods}
\begin{itemize}
\item \verb|findWormLazyCropped()|: 
\item \verb|findWormLazy()|: 
\item \verb|findWormLazyCroppedDemo()|: 
\item \verb|processFrame(img)|: dispatches (see \ref{tab:options})
\end{itemize}


\begin{table}[h]
\begin{tabular}{|p{10cm}|}
\hline
\begin{verbatim} 
options = {
            'test' : self.wormTest,
            'conf' : self.wormTest,
            'lazy' : self.findWormLazy,
            'lazyc': self.findWormLazyCropped,
            'lazyd': self.findWormLazyCroppedDemo,
            'full' : self.findWormFull, #segmentation
            'pca'  : self.findWormPCA, 
            'box'  : self.findWormBox,
            'surf' : self.surfImp, 
            'sift' : self.siftImp,
            'mser' : self.mserImp
            }
\end{verbatim} \\
\hline
\end{tabular}
\caption{processFrame modes (linked to -m) }
\label{tab:options}
\end{table}


%Tavle \ref{tab:options}




\end{document}